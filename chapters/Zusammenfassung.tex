\chapter*{Zusammenfassung}
\thispagestyle{plain2}
\addcontentsline{toc}{chapter}{Zusammenfassung}
\pagenumbering{roman}


DNA-Reparaturmechanismen sind unabdingbar, um Zellen gegen innere und \"{a}u\ss{}ere Einfl\"{u}sse zu sch\"{u}tzen. Dies \"{a}u\ss{}ert sich in einer erh\"{o}hten Anf\"{a}lligkeit f\"{u}r Zellalterung und Krebsentwicklung infolge eines beeintr\"{a}chtigten Reparaturapparats. Der Reparaturprozess wird von enzymatisch aktiven, makromolekularen Komplexen ausgef\"{u}hrt, die sich an gesch\"{a}digten Chromatinfasern assemblieren. Fluoreszenzmikroskopische Untersuchung\-en bez\"{u}glich der Dynamik einzelner Reparaturfaktoren haben gezeigt, dass die Komplexbildung durch einen schnellen Proteinaustausch charakterisiert ist, wohingegen die Reparaturzeit insgesamt wesentlich l\"{a}nger dauert. Inwiefern die schnellen und reversiblen Austauschkinetiken zur Regulation von \"{u}bergeordneten Systemeigenschaften, wie z.B. der Reparaturrate und deren Robustheit, beitragen, ist bisher nur unzureichend verstanden.\\
In dieser Arbeit quantifizieren wir die Beziehung der Proteindynamiken untereinander und ihren Einfluss auf die funktionalen Eigenschaften des DNA-Reparaturprozesses durch zeitaufgel\"{o}ste Messungen der Nukleotidexzisionsreparatur (NER) in S\"{a}ugerzellen in Verbindung mit mathematischer Modellierung. Wir finden heraus, dass es sich bei der DNA Reparatur, ungeachtet der Komplexit\"{a}t der grundlegenden molekularen Prozesse, um eine langsame Reaktion erster Ordnung mit einer Halb\-wertzeit von $\sim$1 Stunde handelt. 
Um einen weitergehenden funktionellen Einblick zu bekommen, entwickeln wir ein ma\-the\-matisches Modell des NER Reaktionsweges und identifizieren anhand experimenteller Daten die Modellparameter. Eine wichtige Vorhersage des Modells ist, dass die Reparatursynthese nicht von einer einzigen ratenlimitierenden Komponente kontrolliert wird. Vielmehr zeigen alle Reparaturfaktoren einheitlich niedrige Kontroll\-koeffizienten und tragen daher gemeinsam zur Kontrolle der Reparaturrate bei. Durch Ausnutzen der nat\"{u}rlichen Proteinexpressionsvariabilit\"{a}t zwischen Einzelzellen k\"{o}nnen wir die Modell\-vorhersage einer geteilten Kontrolle der Reparaturrate f\"{u}r die NER-Faktoren XPC, TFIIH, XPA, XPF und RPA experimentell best\"{a}tigen. Quantitativ stellen wir jedoch eine Diskrepanz zwischen den vorausgesagten und gemessenen Kontrollkoeffizienten fest, wobei letztere insgesamt gr\"{o}\ss{}ere Werte aufweisen. Bei der Suche nach einer m\"{o}glichen Erkl\"{a}rung f\"{u}r diese Diskrepanz beobachten wir eine starke positive Korrelation der nu\-klearen NER-Faktorkonzentrationen, was auf eine Kore\-gulation der Proteinexpression hinweist. Unter Beachtung dieser Koregulation l\"{a}sst sich die Diskrepanz zwischen gemessener und vorausgesagter Reparaturantwort aufl\"{o}sen.\\              
Die in dieser Arbeit gewonnenen Erkenntnisse zeigen zwei komplement\"{a}re Re\-gulationsmechanismen der DNA-Reparatur auf. Zum einen ist die Kontrolle der Reparaturrate unter allen Reparaturfaktoren aufgeteilt, was Fluktuationen in einzelnen NER-Faktorkonzentrationen kompensiert. Zum anderen d\"{a}mmt die Koregulation der NER-Faktorexpression diese Proteinfluktuationen ein und liegt damit m\"{o}glicherweise der koordinierten Regulierung der DNA-Reparaturkapazit\"{a}t zugrunde. Auch andere chromatin\-assoziierte Prozesse, wie z.B. die Transkription, werden durch Multiproteinkomplexe, die sich reversibel an Chromatin assemblieren, katalysiert. Die hier entwickelte mathemati\-sche Beschreibung k\"{o}nnte daher als Ausgangspunkt f\"{u}r ein systemisches Verst\"{a}ndnis der Transkriptionsregulation dienen.




  




