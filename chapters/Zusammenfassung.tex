\chapter*{Zusammenfassung}
\thispagestyle{plain2}
\addcontentsline{toc}{chapter}{Zusammenfassung}
\pagenumbering{roman}


DNA-Reparaturmechanismen sind unabdingbar, um die Zelle gegen innere und \"{a}u\ss{}ere Einfl\"{u}ssen zu sch\"{u}tzen. Dies \"{a}u\ss{}ert sich in einer erh\"{o}hten Anf\"{a}lligkeit f\"{u}r Zellalterung und Krebsentwicklung infolge eines beeintr\"{a}chtigten Reparaturapparats. Der Reparaturprozess wird von enzymatisch aktiven, makromolekularen Komplexen ausgef\"{u}hrt, die sich an gesch\"{a}digten Chromatinfasern assemblieren. Fluoreszenzmikroskopische Untersuchung\-en einzelner Reparaturfaktordynamiken haben gezeigt, dass die Komplexbildung durch einen schnellen Proteinaustausch charakterisiert ist, wohingegen die Gesamtreparaturzeit wesentlich l\"{a}nger ausf\"{a}llt. Inwiefern die schnellen Austauschkinetiken zur Regulation von \"{u}bergeordneten Systemeigenschaften, wie z.B. der Reparaturrate und deren Robustheit, beitragen, ist bisher nur unzureichend verstanden.\\
In dieser Arbeit quantifizieren wir die Beziehung der Proteindynamiken und ihren Einfluss auf die funktionalen Eigenschaften des Reparaturprozesses durch zeitaufgel\"{o}ste Messungen der Nukleotidexzisionsreparatur (NER) in S\"{a}ugerzellen und mathematischer Modellierung. Wir finden heraus, dass es sich bei der DNA Reparatur, ungeachtet der Komplexit\"{a}t der grundlegenden molekularen Prozesse, um eine langsame Reaktion erster Ordnung mit einer Halb\-wertzeit von $\sim$1 Stunde handelt. 
Um einen funktionellen Einblick zu bekommen, entwickelten wir ein ma\-the\-matisches Modell des NER Reaktionsweges und identifizierten anhand experimenteller Daten dessen Parameter. Eine wichtige Vorhersage des Modells ist, dass die Reparatursynthese nicht von einer ratenlimitierenden Komponente kontrolliert wird. Vielmehr zeigen alle Reparaturfaktoren einheitlich niedrige Kontroll\-koeffizienten, und tragen daher gemeinsam zur Kontrolle der Reparaturrate bei. Durch Ausnutzen der nat\"{u}rlichen Proteinexpressionsvariabilit\"{a}t zwischen Einzelzellen konnten wir die Modell\-vorhersage einer verteilten Kontrolle der Reparaturrate f\"{u}r die NER-Faktoren XPC, TFIIH, XPA, XPF und RPA experimentell best\"{a}tigen. Quantitativ stellten wir jedoch eine Diskrepanz zwischen den gemessenen und vorausgesagten mittleren Responsekoeffizienten fest. Auf der Suche nach weiteren Quellen, die zur Kontrolle der Reparaturrate beitragen, beobachteten wir eine starke positive Korrelation der nu\-klearen NER-Faktor-Konzentrationen, was auf eine Ko-Regulierung der Proteinexpression hinweist. Unter Beachtung dieser Ko-Regulierung war es uns m\"{o}glich den Unterschied zwischen gemessener und vorausgesagter Reparaturantwort aufzul\"{o}\ss{}en.\\              
Die hier gewonnenen Erkenntnisse zeigen zwei einander erg\"{a}nzende Regulationsmechanismen der DNA-Reparatur auf. Einerseits ist die Kontrolle der Reparaturrate unter allen Reparaturfaktoren aufgeteilt, was Heterogenit\"{a}ten der Vorkommen einzelner NER-Faktoren kompensiert. Andererseits, werden diese Proteinfluktuationen durch die Ko-Regulierung der NER-Faktorexpression zus\"{a}tzlich einged\"{a}mmt. Da alle chromatinassoziierten Prozesse durch Multiproteinkomplexe, die sich assemblieren und reversibel an Chromatin binden, katalysiert werden, k\"{o}nn\-ten sich diese beiden grundlegenden Mechanismen als allgemeine Kontrollmodi auch f\"{u}r die Transkription erweisen.  


  




