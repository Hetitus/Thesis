\chapter*{Zusammenfassung}
\thispagestyle{plain2}
\addcontentsline{toc}{chapter}{Zusammenfassung}
\pagenumbering{roman}


Die DNA Reparatur ist unabdingbar f\"{u}r den intrazellul\"{a}ren Schutz gegen\"{u}ber sch\"{a}digenden Einfl\"{u}ssen aus dem \"{a}u\ss{}eren und inneren Umfeld der Zelle. Dies \"{a}u\ss{}ert sich in einer erh\"{o}hten Anf\"{a}lligkeit f\"{u}r zellul\"{a}res Altern und Krebsentwicklung infolge eines beeintr\"{a}chtigten Reparaturapparats. Der Reparaturprozess wird dabei von enzymatisch aktiven, makromolekularen Komplexen ausgef\"{u}hrt, die sich an den gesch\"{a}digten Fasern des Chromatins assemblieren. Fluoreszenzabbildungen einzelner Reparaturfaktordynamiken zeigen, dass dieser Vorgang durch einen schnellen Proteinaustausch charakterisiert ist, wohingegen die \"{u}bergreifende Reparaturzeit wesentlich l\"{a}nger ausf\"{a}llt. Inwiefern diese molekularen Interaktionen zur Regulation von \"{u}bergeordneten Systemeigenschaften, wie z.B. der Reparaturrate und deren Robustheit, beitragen, ist bisher nur unzureichend verstanden.\\
In dieser Arbeit quantifizieren wir die Beziehung der Proteindynamiken und ihren Einfluss auf den Reparaturprozess durch die zeitaufgel\"{o}ste Messung der Nukleotidexcisionsreparatur (NER) in S\"{a}ugerzellen. Wir finden heraus, dass es sich bei der DNA Reparatur, ungeachtet der molekularen Komplexit\"{a}t, um eine langsame Reaktion erster Ordnung mit einer Halb\-wertzeit von 1.2 Stunden handelt.\\  
Um einen funktionellen Einblick zu bekommen, entwickeln wir ein identifizierbares ma\-the\-matisches Modell des NER Reaktionsweges. Gem\"{a}\ss{} einer Modellvorhersage wird die Reparatursynthese nicht von einer Raten limitierenden Komponente allein kontrolliert. Stattdessen tragen alle NER Faktoren gemeinsam zur Kontrolle der Reparaturrate bei, was anhand der einheitlich niedrigen Responsekoeffizienten ersichtlich wird. Durch Ausnutzen der nat\"{u}rlichen Zell-Zell-Proteinexpressionsvariabilit\"{a}t k\"{o}nnen wir qualitativ den Fund einer geteilten Kontrolle f\"{u}r die NER-Faktoren XPC, TFIIH, XPA, XPF und RPA experimentell untermauern. Quantitativ ist die gemessene durchschnittliche 'Response' der Reparaturrate jedoch signifikant h\"{o}her als der vorausgesagte mittlere Responsekoeffizient. Auf der Suche nach weiteren Quellen, die zur Kontrolle der Reparaturrate beitragen, beobachten wir eine starke Korrelation der nuklearen NER-Faktor-Konzentrationen, was auf einen komplement\"{a}ren Kontrollmechanismus zur Regulierung dieser Proteinkonzentrationen hinweist. Beachtenswert ist, dass sich die Diskrepanz zwischen gemessener und vorausgesagter Reparaturresponse durch die Verwendung der ermittelten Kreuzkorrelationen in der Kontrollanalyse aufl\"{o}\ss{}t.\\              
Die hier gewonnenen Erkenntnisse portr\"{a}tieren zwei einander erg\"{a}nzende Regulationsmodi der DNA-Reparatur. Einen, wo die gro\ss{}e Zeitskala der Reparaturkomplexassemblierung, generiert durch die Stochastizit\"{a}t der Proteinaustauschdynamiken, Heterogenit\"{a}ten der nu\-klearen Vorkommen einzelner NER-Faktoren kompensiert. Und einen zweiten, mechanistisch unbekannten Modus, der die nukleare NER-Faktorexpression, selbst, kontrolliert. In Anbetracht des sich \"{a}hnelnden dynamischen Designs, k\"{o}nnten sich diese beiden Ans\"{a}tze als \"{u}bertragbar auch auf andere Chromatin assoziierte Prozesse, wie z.B. die Transkription oder Replikation, erweisen.  


  




