\chapter*{Zusammenfassung}
\thispagestyle{plain2}
\addcontentsline{toc}{chapter}{Zusammenfassung}
\pagenumbering{roman}


Die DNA Reparatur ist unabdingbar f\"{u}r den intrazellul\"{a}ren Schutz gegen\"{u}ber sch\"{a}digenden Einfl\"{u}ssen aus dem \"{a}u\ss{}eren und inneren Umfeld der Zelle. Dies \"{a}u\ss{}ert sich in einer erh\"{o}hten Anf\"{a}lligkeit f\"{u}r zellul\"{a}res Altern und Krebsentwicklung infolge eines eingschr\"{a}nkten Reparaturapparats. Der Reparaturprozess wird dabei von enzymatisch aktiven, makromolekularen Komplexen ausgef\"{u}hrt, die sich an den gesch\"{a}digten Fasern des Chromatin assemblieren. Fluoreszenzabbildungen einzelner Reparaturfaktor-Dynamiken zeigen, dass dieser Vorgang durch einen schnellen Proteinaustausch charakterisiert ist, wohingegen die \"{u}bergreifende Reparaturzeit wesentlich l\"{a}nger ausf\"{a}llt. Inwiefern diese molekularen Interaktionen zur Regulation von \"{u}bergeordneten Systemeigenschaften beitragen, ist bisher nur unzureichend verstanden.\\
In dieser Arbeit quantifizieren wir die Beziehung der Proteindynamiken und ihren Einfluss auf den Reparaturprozess durch die zeitaufgel\"{o}ste Messung der Nukleotidexcisionsreparatur (NER) in S\"{a}ugerzellen. Wir finden heraus, dass es sich bei der DNA Reparatur, ungeachtet der molekularen Komplexit\"{a}t, um eine langsame Reaktion erster Ordnung mit einer Halbwertzeit von 1.2 Stunden handelt.\\  
Um einen funktionellen Einblick zu bekommen, entwickeln wir ein identifizierbares mathematisches Modell des NER Reaktionswegs. Gem\"{a}\ss{} einer Modellvorhersage wird die Reparatursynthese nicht von einer Raten limitierenden Komponente allein kontrolliert. Ersichtlich anhand der einheitlich niedrigen Responsekoeffizienten, tragen stattdessen alle NER Faktoren gemeinsam zur Kontrolle der Reparaturrate bei. Durch Ausnutzen der nat\"{u}rlichen Zell-Zell-Proteinexpressionsvariabilit\"{a}t k\"{o}nnen wir den Fund einer geteilten Kontrolle f\"{u}r die NER-Faktoren XPC, TFIIH, XPA, XPF und RPA qualitativ experimentell untermauern. Quantitativ ist die gemessene durchschnittliche 'Response' der Reparaturrate jedoch signifikant h\"{o}her als der vorausgesagte mittlere Responsekoeffizient. Auf der Suche nach weiteren Quellen, die zur Kontrolle der Reparaturrate beitragen, beobachten wir eine starke Korrelation der nuklearen NER Faktor Konzentrationen. Dies weist auf einen komplement\"{a}ren Kontrollmechanismus zur Regulierung der NER-Faktorkonzentrationen hin. Beachtenswert ist, dass das             



%  Notably, including the identified cross-correlation in the control analysis resolves the discrepancy between measured and predicted repair-rate response.\\  
%These findings portray two complementary modes of robust regulation in DNA repair. One, where the long repair-complex assembly time-scale, generated by the stochastic nature of the protein exchange dynamics, compensates for heterogeneities in the nuclear abundance of single repair proteins. And a second, mechanistically unknown mode controlling the nuclear NER factor expression itself. Given the similar dynamical design, these may also be valid approaches in other chromatin-associated processes such as transcription or replication.



