\chapter*{Zusammenfassung}
\thispagestyle{plain2}
\addcontentsline{toc}{chapter}{Zusammenfassung}
\pagenumbering{roman}


DNA-Reparaturmechanismen sind unabdingbar, um die Zelle gegen innere und \"{a}u\ss{}ere Einfl\"{u}ssen zu sch\"{u}tzen. Dies \"{a}u\ss{}ert sich in einer erh\"{o}hten Anf\"{a}lligkeit f\"{u}r Zellalterung und Krebsentwicklung infolge eines beeintr\"{a}chtigten Reparaturapparats. Der Reparaturprozess wird von enzymatisch aktiven, makromolekularen Komplexen ausgef\"{u}hrt, die sich an gesch\"{a}digten Chromatinfasern assemblieren. Fluoreszenzmikroskopische Untersuchung\-en einzelner Reparaturfaktordynamiken haben gezeigt, dass die Komplexbildung durch einen schnellen Proteinaustausch charakterisiert ist, wohingegen die Gesamtreparaturzeit wesentlich l\"{a}nger ausf\"{a}llt. Inwiefern die schnellen Austauschkinetiken zur Regulation von \"{u}bergeordneten Systemeigenschaften, wie z.B. der Reparaturrate und deren Robustheit, beitragen, ist bisher nur unzureichend verstanden.\\
In dieser Arbeit quantifizieren wir die Beziehung der Proteindynamiken und ihren Einfluss auf den Reparaturprozess durch die zeitaufgel\"{o}ste Messung der Nukleotidexzisionsreparatur (NER) in S\"{a}ugerzellen. Wir finden heraus, dass es sich bei der DNA Reparatur, ungeachtet der Komplexit\"{a}t der grundlegenden molekularen Prozesse, um eine langsame Reaktion erster Ordnung mit einer Halb\-wertzeit von 1.2 Stunden handelt. 
Um einen funktionellen Einblick zu bekommen, entwickeln wir ein identifizierbares ma\-the\-matisches Modell des NER Reaktionsweges. Eine wichtige Vorhersage des Modells ist, dass die Reparatursynthese nicht von einer ratenlimitierenden Komponente allein kontrolliert wird. Vielmehr zeigen alle Reparaturfaktoren einheitlich niedrige Kontrollkoeffizienten, und tragen daher gemeinsam zur Kontrolle der Reparaturrate bei. Durch Ausnutzen der nat\"{u}rlichen Proteinexpressionsvariabilit\"{a}t zwischen Einzelzellen k\"{o}nnen wir die Modell\-vorhersage einer verteilten Kontrolle der Reparaturrate f\"{u}r die NER-Faktoren XPC, TFIIH, XPA, XPF und RPA experimentell qualitativ best\"{a}tigen. Quantitativ ist die gemessene durchschnittliche Antwort der Reparaturrate jedoch signifikant h\"{o}her als der vorausgesagte mittlere Responsekoeffizient. Auf der Suche nach weiteren Quellen, die zur Kontrolle der Reparaturrate beitragen, beobachten wir eine starke positive Korrelation der nu\-klearen NER-Faktor-Konzentrationen, was auf einen komplement\"{a}ren Kontrollmechanismus zur Regulierung dieser Proteinkonzentrationen hinweist. Beachtenswert ist, dass sich die Diskrepanz zwischen gemessener und vorausgesagter Reparaturantwort aufl\"{o}\ss{}t, wenn die Kontrollanalyse um die gemessenen Kreuzkorrelationen erweitert wird.\\              
Die hier gewonnenen Erkenntnisse zeigen zwei einander erg\"{a}nzende Regulationsmechanismen der DNA-Reparatur auf. Im ersten Mechanismus generiert die Stochastizit\"{a}t der Proteinaustauschdynamiken eine lange Zeitskala, die Heterogenit\"{a}ten der nu\-klearen Vorkommen einzelner NER-Faktoren kompensiert. Im zweiten Mechanismus, dessen molekulare Details unbekannt sind, wird die NER-Faktorexpression selbst kontrolliert. Da chromatinassoziierte Prozesse wie z.B. Chromatinremodellierung, Transkription und Translation ein \"{a}hnliches dynamische Design wie die DNA Reparatur besitzen, k\"{o}nn\-ten sich diese beiden grundlegenden Mechanismen als allgemeine Kontrollmodi f\"{u}r Chromatindynamiken erweisen.  


  




