\chapter*{Zusammenfassung}
\addcontentsline{toc}{chapter}{Zusammenfassung}
\pagenumbering{roman}
The nucleotide-excision repair pathway removes mutagen-inflicted DNA lesions from the genome. Repair proteins recognize DNA lesions and form multi-protein complexes that catalyze the excision of the lesion and the re-synthesis of the excised part. Imaging the dynamics of fluorescently labeled repair proteins in living human cells has revealed that all factors continuously and rapidly exchange at repair sites. We asked how this dynamic mode of protein-complex assembly shapes the repair process. Measuring repair DNA synthesis in intact cells, we obtained a surprisingly simple result. Over the entire process, the rate is proportional to the amount of DNA lesions, where the proportionality factor is a single 'slow' rate constant. Such kinetic behavior is often regarded as evidence for a rate-limiting step, but we show here that it is an emergent property of the dynamic interplay of many repair proteins. As a consequence, the rate of DNA repair is a systems property that is controlled collectively by the expression levels of all repair factors. Given that transcription in living cells has similar dynamic features - rapidly exchanging components of the transcription machinery and slow bursts of mRNA synthesis - collective rate control might be a general property of chromatin-associated molecular machines.


