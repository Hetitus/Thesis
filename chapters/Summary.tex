\chapter*{Summary}
\addcontentsline{toc}{chapter}{Summary}





DNA repair is indispensable for the intracellular protection against environmental and endogenous damaging agents.
This is reflected in an increased susceptibility against cellular aging and cancer development as a consequence to impaired repair. 


%Functional repair is carried out by enzymatic macromolecular complexes that assemble at specific sites on the chromatin fiber. How the rate of these molecular machineries is regulated by their constituent parts is poorly understood. 
%Here we quantify nucleotide-excision DNA repair (NER) in mammalian cells and find that, despite the pathways' molecular complexity, repair effectively obeys slow first-order kinetics. Theoretical analysis indicates that these kinetics are not due to a singular rate-limiting step. Rather, first-order kinetics emerge from the interplay of rapidly and reversibly assembling repair proteins, stochastically distributing DNA lesion repair over a broad time period. Based on this mechanism, the model predicts that the repair proteins collectively control the repair rate. Exploiting natural cell-to-cell variability, we corroborate this prediction for the lesion-recognition factors XPC and XPA. 
%Our findings provide a rationale for the emergence of slow time scales in chromatin-associated processes from fast molecular steps and suggest that collective rate control might be a widespread mode of robust regulation in DNA repair and transcription.
%harnessing


%\end{document}