\chapter*{Summary}
\thispagestyle{plain2}
\addcontentsline{toc}{chapter}{Summary}





DNA repair is indispensable for the protection of cells against environmental and endogenous damaging agents. This is reflected in an increased susceptibility to cellular ageing and cancer development as a consequence of impaired repair. The repair process is carried out by enzymatic macromolecular complexes assembling at damaged sites on the chromatin fibre. Fluorescence imaging of the individual repair factor dynamics shows a rapid protein exchange at the repair sites, while the overall repair process takes significantly longer. How these fast and reversible molecular interactions regulate the emergence of systems-level properties, such as the repair rate and its robustness, is poorly understood.\\ 
In this thesis, we quantify the relation between the protein dynamics and the resulting functional properties of DNA repair by time-resolved measurements of nucleotide-excision DNA repair (NER) in mammalian cells and mathematical modelling. We find that despite the pathway's molecular complexity, DNA repair follows a slow first-order reaction with a half-time of $\sim$1 hour. To gain functional insights, we develop a mathematical model of the NER pathway and identify its parameters from experimental data. According to our model predictions, repair synthesis is not controlled by a rate-limiting component. Instead, all NER factors collectively control the repair-rate, as indicated by uniformly small response coefficients. Harnessing the natural cell-to-cell variability of protein expression, we provide experimental support for shared rate control by the core NER factors XPC, TFIIH, XPA, XPF and RPA. However, quantitatively we find a discrepancy between the predicted and measured response coefficients, with the latter being overall larger. Probing into the causes for this discrepancy, we discover that the expression of nuclear NER factors is strongly correlated, suggesting a co-regulation of protein expression. Notably, we are able to resolve the difference between measured and predicted repair-rate response by accounting for this co-regulation. \\ 
These findings portray two complementary modes of robust regulation in DNA repair. Firstly, the repair-rate control is shared among all NER factors, which compensates for fluctuations of the protein concentrations. Secondly, the co-regulation of nuclear NER factor expression limits these heterogeneities and might underlie the coordinated regulation of the DNA repair capacity. Other chromatin-associated processes such as transcription are also catalysed by multi-protein complexes that reversibly assemble on chromatin. The quantitative framework developed here for NER might therefore also serve as a starting point for dissecting the regulation of transcription at the systems level.



%provide a rationale for the emergence of slow time scales in chromatin-associated processes from fast molecular steps and suggest that collective rate control might be a widespread mode of robust regulation in DNA repair and transcription.
      
 


%\end{document}