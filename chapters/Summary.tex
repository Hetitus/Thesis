\chapter*{Summary}
\thispagestyle{plain2}
\addcontentsline{toc}{chapter}{Summary}





DNA repair is indispensable for the intracellular protection against environmental and endogenous damaging agents. This is reflected in an increased susceptibility to cellular ageing and cancer development as a consequence to impaired repair. The repair process is carried out by enzymatic macromolecular complexes assembling at damaged sites on the chromatin fibre. Fluorescence imaging of the individual repair factor dynamics show a rapid protein exchange at the repair sites, while the overall repair time is significantly longer. How these fast molecular interactions regulate the emergence of higher-level system-properties, such as the repair rate and its robustness, is poorly understood.\\ 
In this thesis we quantify the relation between the protein dynamics and the resulting repair by measuring the time course of nucleotide-excision DNA repair (NER) in mammalian cells. We find that, despite the pathway's molecular complexity, DNA repair follows a slow first-order reaction with a half-time of 1.2 hours. \\
To gain functional insight, we develop an identifiable mathematical model of the NER pathway. According to a model prediction repair synthesis is not controlled by one rate-limiting component alone. Instead, all NER factors collectively control the repair-rate, which is indicated by uniformly small response coefficients. Harnessing the natural cell-to-cell variability of protein expression, we show qualitative experimental support for the shared control of the core NER factors XPC, TFIIH, XPA, XPF and RPA. However, quantitatively the measured average repair-rate response is significantly larger than the predicted mean response coefficient. In search of additional sources contributing to the repair-rate control, we observe that the nuclear NER factor expression is strongly correlated suggesting a complementary control mechanism regulating NER factor expression values. Notably, including the identified cross-correlation in the control analysis resolves the discrepancy between measured and predicted repair-rate response.\\  
These findings portray two complementary modes of robust regulation in DNA repair. One, where the long time-scale of the repair-complex assembly, generated by the stochastic nature of the protein exchange dynamics, compensates for heterogeneities in the nuclear abundance of single repair proteins. And a second, mechanistically unknown mode controlling the nuclear NER factor expression itself. Given the similar dynamical design, these may also be valid approaches in other chromatin-associated processes such as transcription or replication.    



%provide a rationale for the emergence of slow time scales in chromatin-associated processes from fast molecular steps and suggest that collective rate control might be a widespread mode of robust regulation in DNA repair and transcription.
      
 


%\end{document}