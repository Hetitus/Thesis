\chapter{Discussion}


Our DNA-repair analysis was made possible by the ability to explicitly observe the kinetics of resynthesised DNA during nucleotide-excision repair (NER). The quantification of this repair intermediate by following the incorporation of fluorescently labelled nucleotides on the single cell level shows that the repair process follows a slow, first-order kinetic (\textit{cf.}\ Chapter \ref{chap:quantData}). These new data supplement comprehensive fluorescent real-time measurements quantifying the dynamic behaviour of single repair proteins. Hence, we can observe the individual protein exchange kinetics at the chromatin fibre, which eventually lead to the formation of catalytic multi-protein machineries, concomitantly with the global repair process. Based on this information we were able to identify a detailed kinetic model of NER, which allows us to establish a quantitative link between fast and random assembly of the NER factors at the DNA template and the emergent phenomenon of a slow overall repair time (\textit{cf.}\ Chapter \ref{chap:kineticNERmodel}). The most important implication of the model is that the repair rate is controlled by the repair factors in a collective way. In particular, the uniformly small response coefficients emphasize the robustness of the DNA repair process. This disproves the widely-held assumption that a single rate-limiting step controls the entire repair process \cite{Hoogstraten2008,Koberle2006,Kang2011}. Our theory of a shared control has been corroborated qualitatively by exploiting experimental measurements of the natural variability of the nuclear NER factor concentrations (\textit{cf.}\ Chapter \ref{chap:robustRepair}). However quantitatively, the measured repair responses exceed the magnitude of the predicted response coefficients significantly.\\
While further pursuing the nature of the measured cell-to-cell variability and its influence on DNA repair, we observed that the substantial heterogeneity of the repair rate is generated primarily by the distribution of inflicted DNA lesions. This factor is complemented by the variability contributed from the nuclear concentration of each individual repair factor. The quantitative impact of these two parameters on the cell-to-cell variability of DNA repair is consistent with the robustness of the repair rate against concentration fluctuations of individual repair factors (\textit{cf.}\ Chapter \ref{chap:robustRepair}).\\  
Interestingly, we also found evidence for the regulated NER factor expression indicated by strong cross-correlations of the protein concentrations. Considering these mutual interdependencies in our control analysis quantified their contribution to the repair-rate response and thus, explains the gap between measured and predicted DNA repair response (\textit{cf.}\ Chapter \ref{chap:crossCorell}). These results imply that the robustness of the repair rate is not only controlled by the pathway-intrinsic molecular interplay of chromatin-associated machineries but also extrinsically by a regulatory mechanism orchestrating NER factor expression.\\
In the following, we will elaborate a more detailed discussion about individual findings including: The DNA repair kinetics, NER model identifiability, model parametrization, repair-rate control and the co-regulation of NER factor expression.     
% * <l.adlung@dkfz.de> 2014-12-10T23:05:46.669Z:
%
%  Regarding collective control: Do you mention at any point the phenotype of knockouts of the individual repair factors? I remember that you indicating it when speaking about knock down or overexpression...
%

\paragraph{Slow first order kinetics as a general systems property for chromatin associated processes}
In our analysis of the NER kinetic we focused on the removal of one of the major UV-induced DNA lesions, the 6-4 pyrimidine-pyrimidone photoproducts (6-4PPs) (\textit{cf.}\ Figure \ref{fig:DNArepairKinetic}B), which are repaired significantly faster ($\sim$2-3 hours) compared to cyclobutane-pyrimidine dimers (CPDs), for which the majority is still present after $\sim$24 hours \cite{Smerdon1978,vanHoffen:1995:EMBO-J:7835346,Luijsterburg2010}. 6-4PPs are repaired by incision of the damaged DNA strand, and subsequent DNA resynthesis, which we were able to monitor by following the incorporation of the fluorescently tagged thymidine analogue EdU. The nucleotides are labelled by copper-catalysed covalent addition of the fluorophore, which requires the cells to be fixed. Hence, EdU accumulation cannot be measured continuously in one cell. Instead, we followed EdU incorporation stepwise by permeabilising cells after increasing time intervals, each starting immediately after UV-irradiation. In this way we were able to show that the 6-4PP repair fits to a first order kinetic with a half-time of $\sim$1.2 hours.         
This long time-scale is in contrast to the rapid exchange behaviour of early and late repair factors as reported by several live-cell imaging studies. \cite{Houtsmuller1999,Volker2001,Hoogstraten2002,Rademakers2003,Mone2004,Zotter2006,Hoogstraten2008,Luijsterburg2010}. \\
%Recently, these results could be generalized for all core NER factors combining different photobleaching and fluorescence life-time microscopy approaches \cite{}.The work in this thesis strengthens a prominent explanation of this time-scale contradiction using the comprehensive biological knowledge about NER together with mathematical modelling.\\
In agreement with a simplified model simulating the assembly of a catalytic protein repair-complex \cite{Terstiege2010}, we find here that reversible exchange of NER factors is the key property that makes NER an apparent first-order process. In this description, the long time-scale arises from the concurrence of stochastic assembly and reversible exchange of the constituent components, which leads to the formation of many non-functional preliminary stages before eventually the catalytic complex is completed. This is in contrast to a sequence of irreversible binding steps, which will create sigmoidal kinetics with an even sharper delay with increasing numbers of assembling components \cite{Terstiege2010}. As suggested by Luijsterburg \textit{et al.} (2010) \cite{Luijsterburg2010}, the slow repair time might actually present an advantage for the specificity of the repair process, analogous to a kinetic proof-reading mechanism, where differential dissociation kinetics would prevent false-positive lesion detection. \\ 
The phenomenon of rapid protein exchange is widespread also in other chromatin associated processes such as replication, chromatin remodelling or transcription \cite{McNairn2005,Erdel2011,Sonneville2012,Stasevich2011}. Analogous to NER, these processes involve finding a specific reaction site before the catalytic machinery can assemble at the same location. In the particular case of DNA transcription, transient protein dynamics were experimentally demonstrated for transcription factor binding and transcription machinery assembly \cite{Hager2009}. The time scales for these reactions (several seconds to minutes) are comparable to our results for DNA repair. This encourages us to hypothesise that the slow time scales of transcriptional bursting in mammalian cells (tens of minutes to hours; \cite{Harper2011,Suter2011}) also arise through the reversible assembly of large macromolecular complexes.   



\paragraph{A predictive kinetic model of DNA repair}
Augmenting the extensive kinetic data on binding and dissociation of individual components of the NER machinery \cite{Luijsterburg2010} with a direct readout for DNA repair synthesis, we were able to develop a predictive ODE-model of NER (\textit{cf.}\ Chapter \ref{chap:kineticNERmodel}). To the best of our knowledge it is the first model of a DNA-associated process, for which the model parameters could be identified (\textit{i.e.}, parameter values uniquely assigned with narrow confidence bounds) from experimental data. \\
On the technical side, to make the profile likelihood estimation for each parameter computationally feasible, we first needed to optimize the runtime of the parameter estimation algorithm. Therefore, we integrated the model into a dynamic modelling framework that applies a deterministic trust-region fitting algorithm implemented in MATLAB \cite{Raue2009}. The accuracy and computational performance of the implementation benefit immensely from the simultaneous integration of the dynamic variables concomitantly with the derivatives of the objective function with respect to the model parameters \cite{conn2009introduction,Ramachandran2010,Raue2013}. Solving this extended system of ordinary differential equations with a CVODE solver implemented in C accelerated the speed for one deterministic optimization run by 200-fold compared to the previously applied stochastic Markov Chain Monte Carlo (MCMC)\label{sec:MCMC} optimization algorithm \cite{Terstiege2010}. The runtime spent for one fit on a 12-core processor with 3 GHz each is now reduced from two weeks to $\sim$10 minutes. This considerable runtime improvement allowed us to sample the parameter space exhaustively using Latin hypercube sampling, a technique, which makes the deterministic optimization algorithm more robust against local optima \cite{Raue2013}.\\
In addition to fitting the parameters that determine the model dynamics, we estimated the parameters that characterize the measurement noise. This approach of explicitly including the measurement noise, has the advantage that it avoids an inaccurate \textit{ad hoc} estimate of the experimentally-determined error due to low numbers of replicates and thus gives a more exact assessment of the model parameters, without preprocessing the experimental data \cite{Raue2013}. \\
Imposed by the non-identifiability of several parameters in the model of Luijsterburg \textit{et al.} \cite{Luijsterburg2010}, a number of simplifications concerning the model structure were required to achieve an identifiable model of the NER pathway. The two most significant modifications that we introduced here are the removal of the 'partially unwound' repair intermediate and the neglect of protein-protein interactions, in particular the sequential binding of XPA and XPF \cite{Volker2001}, for which there was no evidence in the data. Since we lack a direct readout for incised DNA, we additionally assumed a practically instantaneous incision of the DNA lesion once the pre-incision complex has been completely assembled. The missing information about DNA incision also limits the identifiability of the other catalytic rates (despite the rate of rechromatinisation), indicated by the existence of lower confidence bounds, only. This case of structural non-identifiability proved to be without consequence for the goodness of the model predictions (\textit{cf.}\ Figure \ref{fig:controlCoefficients}). Moreover, fast catalytic rate constants agree with a recent direct measurement in bacteria where DNA synthesis and ligation take seconds \cite{Uphoff2013}. \\
Our current model, whose structure is reconcilable with the experimental data, constitutes a concrete improvement over previous NER models \cite{Luijsterburg2010,Politi2005,Kesseler2007}. We believe that, gaining further mechanistic understanding about system properties requires an intertwined advancement, where the addition of molecular detail in the model is consecutively balanced by appropriate quantitative measurements. Already at the present level of detail, we are confident that the general dynamic behaviour of the model (\textit{e.g.} repair rate, robustness against protein-expression noise and fidelity of lesion recognition) will prove robust with respect to such a development as we observed these properties already for a simplified model of repair (\textit{cf.}\ Figure \ref{fig:reactionTiming} \cite{Verbruggen2014}).    
       

\paragraph{Model parametrization and its implications}
The model fit (described in detail in Chapter \ref{chap:kineticNERmodel}) yields biochemically plausible estimates for the kinetic parameters of the individual assembly and dissociation kinetics \textit{in vivo}, which account for both the long-term accumulation and for the rapid exchange of the NER factors. Notably, the affinity of the lesion recognition factor XPC is remarkably low, which is mainly caused by the high off-rate. This is consistent with previous work where high molecular off-rates have been described as a general property of self-organizing systems, which allows for efficient exploration of an assembly landscape and selection of a functional steady state \cite{Kirschner2000}. As discussed in depth at Luijsterburg and coworkers (2010)\cite{Luijsterburg2010} the strong reversibility is also beneficial for the specificity and regulation of the system, for example by preventing the trapping of NER proteins in incomplete (and thus unproductive) repair complexes. The reversibility advantage also applies for the trade-off between specificity and efficiency of the mechanism, which equally plays a role in chromatin remodelling and transcription \cite{Cook2010,Voss2011}.\\
Apart from fast reversible assembly/dissociation, parameter estimation also revealed that the catalytic rate constants are fast, which indicates a tight coupling between lesion excision and repair synthesis. This in turn implies that the lesions are repaired without much delay immediately after their recognition by XPC, which consequently prevents the accumulation of incised DNA. Our \textit{in-vivo} finding stays in contrast to \textit{in-vitro} experiments that have found a delay before repair synthesis \cite{Mocquet2008,Riedl2003}. Our prediction that the repair system prevents single-stranded repair intermediates to accumulate corresponds with our expectation that an elevated abundance of incised DNA would trigger DNA degradation to avoid the risk of an inflammatory response, autoimmunity \cite{Takeuchi2010} or simply their collapse to double strand breaks \cite{Mocquet2008}.             



\paragraph{Collective control of the DNA repair rate}
A consequence of the NER model architecture, which describes the sequential organization of this chromatin-associated process by cycles of protein recruitment, is the absence of any rate-limiting factor for repair synthesis. Instead, the control of the repair speed is homogeneously distributed, with small contributions from each repair factor. This central result of our work agrees with the outcome of previous flux control analyses of various biochemical pathways that found no experimental support for the existence of a unique rate-limiting factor \cite{Fell1992,Bruggeman2007,Yi2000a}. Moreover, response coefficients with values much smaller than 1 indicate that DNA repair is robust  against natural fluctuations of the protein concentrations in the cell \cite{Bluthgen2013}. It is important to note that the degree of control as a kinetic property of a particular NER factor is independent of its position in the order of engagement in the repair process. Thus, the lesion recognition factor XPC, which binds first, has the same control as XPA or RPA that bind much later to protect the unwound DNA single-strands.\\
As a biochemical network can never be insensitive against all possible perturbations \cite{Bluthgen2013,Csete2002}, DNA repair will be affected by larger fluctuations in the concentration of NER factors. In this respect, the model predicts that the rate of repair eventually drops to zero in case of a strong concentration reduction of any repair factor (\textit{cf.}\ Figure \ref{fig:R_largeProteinVariation}). This agrees with various studies showing that a significant reduction in nuclear XPC and XPA levels leads to decreased cell survival and/or decreased lesion removal \cite{Koberle1999,Koberle2006,Renaud2011}. In contrast, increased levels of ERCC1 and XPA lead to increased resistance in certain tumours against cisplatin, which induces DNA interstrand crosslinks specifically removed by NER \cite{Koberle1999,Koberle2006,Renaud2011,Stewart2007,Arora2010}.   
\\ 
To analyse the effect of natural changes in NER factor expression on the repair rate experimentally, instead of changing the protein expression by gene knock-down or overexpression, which usually cause the partial or complete disruption of essential system functions \cite{Moriya2006}, we exploited the natural variability of the nuclear protein concentration. Despite the weak correlation in the data, we saw a significantly positive dependency between protein expression for all five measured NER factors and DNA resynthesis (\textit{cf.}\ Figure \ref{fig:Nuc_vs_DNAsynthesis}). The regression slopes were evenly distributed and below unity, which corroborates the finding of a shared moderate repair-rate control.\\  
Not surprisingly, there are studies indicating that the collective control of a systems property (\textit{e.g.} repair rate) is not an exclusive concept for the DNA repair pathway, but also applies to other chromatin-associated processes. For example, it was shown that the probability to express interferon-$\beta$ was significantly increased, when five out of six transcription factors necessary for transcriptional activation were overexpressed \cite{Apostolou2008}. A similar result was obtained recently, which suggested that the expression of of IL-2 is induced by the concerted interplay of four transcription factors together with FOXP3 \label{sec:FOXp3} \cite{Bendfeldt2012}. These findings emphasize that also transcription is a collectively controlled process.   

% * <l.adlung@dkfz.de> 2014-12-10T23:28:41.870Z:
%
%  Are those collective overexpression results always compared to single overexpression?
%
%Each cell is exposed to many strong perturbations on

\paragraph{Co-regulation of the nuclear NER factor expression}    The question how robustness to perturbations of the cellular state emerges from the underlying biochemical networks, has attracted considerable interest. A range of regulatory motifs have been proposed as producers of a robust response (\textit{e.g.} negative feedback, incoherent feed-forward loops and functional redundancy) \cite{Alon2007,Bleris2011,Yi2000,Manuscript2011}. However, most of these functional building blocks are not applicable for promoting robust DNA repair as the NER pathway shows robust regulation without any reported signalling events (as described in this work), nor should transcriptional regulation play a role on the considered time scale between damage infliction and repair initiation \cite{Mone2001}. The individual repair reactions purely depend on the repair complex assembly time and thus are determined by the NER factor binding affinities (this work and \cite{Luijsterburg2010}). In essence, robustness of the repair speed relies on the fact that minor delays or speed-ups caused by fluctuations of one repair factor during the assembly process of the full complex, are negligible on the global time scale. Therefore, randomness in many reversible steps is averaged by the law of large numbers. In this way the robustness of DNA repair is directly linked to the dynamic design of the repair pathway.\\     
Intriguingly, we found experimental evidence for the regulated expression of the NER factors themselves, indicated by a strongly positive pairwise correlation for all seven measured repair factor combinations. In contrast to previous studies showing correlated mRNA \label{sec:mRNA} levels for different NER factors in cancers cells \cite{Damia1998,Cheng2000}, we observed this cross-correlated expression in single cells of various types under unperturbed, NER-independent conditions. Despite the overall increase in nuclear NER factor concentrations during S-phase and G2 phase, the cross-correlation persists independently of the cell cycle. Together, these results imply an additional regulative mechanism that indirectly controls the repair rate already on the transcriptional level. Since, as we show, NER control is distributed, we predict that this transcriptional control must act on many factors simultaneously to be effective.\\
Remarkably, the co-expression of NER factors has strong conceptual similarity to bacterial signal transduction systems, where different regulatory enzymes are encoded on the same operon \cite{Kollmann2005,Lovdok2009}. This co-localisation leads to co-expression, which in turn provides concentration robustness against variability of other pathway components \cite{Bluthgen2013}. So far such a transcriptional coupling mechanism is unknown for the repair proteins involved in NER. Therefore it would be interesting to investigate whether this co-expression can be perturbed pre-transcriptionally to further study its impact on the rate of repair. 

% * <l.adlung@dkfz.de> 2014-12-10T23:33:35.858Z:
%
%  Perturbation in the sence of mono-cistronic mRNA?
%
% ^ <l.adlung@dkfz.de> 2014-12-10T23:36:20.513Z:
%
%  Is there any RNAseq data to corroborate correlated mRNA levels for the NER factors?
%

\paragraph{Concluding remarks}
The accurate repair of DNA is essential for the long-term survival of the cell since it is absolutely necessary for the fidelity of all chromatin-associated processes. It is therefore not surprising to find two molecular cooperating mechanisms that ensure robustness of the repair rate against natural variability. First, as a consequence of the dynamic design of the repair complex assembly, there exists a uniformly distributed control of the repair rate, where each factor has only a moderate effect on its own. This translates into a low system response in the event of fluctuations of the NER factor concentrations. Second, a yet unknown extrinsic mechanism regulates the coordinated NER factor expression, which limits the degree of protein variability, presumably, already pre-transcriptionally. Given that chromatin remodelling, transcription and translation have similar dynamic features as DNA repair, both mechanisms may function as a widespread mode of robust regulation of chromatin-associated processes. 


%- was passiert am Anfang wenn Chromatin remodeled wird...;was passiert, wenn zusätzliche Schritte mit einbezogen werden; was ist generell wichtiger für die Zelle - eine schnelle Reparatur oder, dass der Prozess robust ist...; eine langsame Reparatur ermöglicht natürlich auch Kinetik proofreading; wie anfällig ist der Prozess von korrekter Chromatin Remodulierung? Talk about that this is a leap forward that we directly measure DNA synthesis 

%wie kann man das Transkriptionsgefüge destabilisieren... Findet die Regulation der Proteinkonzentration vor oder nach dem ablesen statt? Is there any RNAseq data? mono-cistronic DNA 