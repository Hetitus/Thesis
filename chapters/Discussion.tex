\chapter{Discussion}


Our DNA-repair analysis was motivated by the ability to explicitly observe the kinetics of resynthesized DNA during nucleotide-excision repair. The quantification of this repair intermediate by following the incorporation of fluorescently labelled nucleotides on the single cell level shows the repair process to be a slow, first-order kinetic (cf.\ Chapter \ref{chap:quantData}). The new data supplements our comprehensive knowledge about the dynamic behaviour of single repair proteins comprising their individual exchange kinetics at the chromatin fibre, which eventually lead to the formation of catalytic multi-protein machineries. Based on this information we were able to identify a realistic kinetic model of NER, which established a quantitative link between fast and random assembly of the NER factors at the DNA template and the emergent phenomenon of a slow overall repair time (cf.\ Chapter \ref{chap:kineticNERmodel}). Most importantly, the model predicts a collective control of the repair rate by the repair factors, which rebuts the presumption of a rate-limiting step in the pathway. This finding has been corroborated experimentally exploiting the natural variability of the nuclear NER factor concentrations, which emphasizes the robustness of DNA repair against fluctuations in NER factor expression (cf.\ Chapter \ref{chap:robustRepair}).\\  
In fact, we observe that the substantial heterogeneity in repair synthesis is generated primarily by the distribution of inflicted DNA lesions. This factor is complemented by the variability contributed from each individual repair factor as well es by mutual dependencies in NER factor expression. The quantitative impact of these three parameters on the cell-to-cell variability of the repair rate is consistent with the robustness of the repair rate against concentration fluctuations of individual repair factors (cf.\ Chapter \ref{chap:crossCorell}).
% * <l.adlung@dkfz.de> 2014-12-10T23:05:46.669Z:
%
%  Regarding collective control: Do you mention at any point the phenotype of knockouts of the individual repair factors? I remember that you indicating it when speaking about knock down or overexpression...
%

\paragraph{Slow first order kinetics as a general systems property for chromatin associated processes.}
During our analysis of the NER kinetic we focused on the removal of one of the major UV-induced DNA lesions, the 6-4PPs (cf.\ Figure \ref{fig:DNArepairKinetic}B), which are repaired significantly faster ($\sim$2-3 hours) compared to CPDs, which are still present after $\sim$ 24 hours \cite{Smerdon1978,vanHoffen:1995:EMBO-J:7835346,Luijsterburg2010}. 6-4PPs are cut out by incision of the damaged DNA strand, which we were able to monitor by following the incorporation of the fluorescently tagged uracil analogue EdU. The nucleotides are labelled by coper-catalysed covalent addition of the fluorophore for which the cells have to be fixed. Hence, EdU accumulation cannot be measured continuously in one cell. Instead, we followed EdU incorporation stepwise in increasingly longer time intervals, each starting immediately after UV-irradiation. In this way we were able to quantify the 6-4PP repair time course which fits to a first order kinetic with a half-life of $\sim$ 1.2 hours.         
This long time scale (hours) stays in striking contrast to several live-cell imaging studies, which determined the time of engagement on damaged DNA for early and late repair factors in the range of minutes \cite{Houtsmuller1999,Volker2001,Hoogstraten2002,Rademakers2003,Mone2004,Zotter2006,Hoogstraten2008}. Recently, this result could be generalized for all core NER factors combining different photobleaching and fluorescence life-time microscopy approaches \cite{Luijsterburg2010}.\\
The work in this thesis strengthens a prominent explanation of this time-scale contradiction using the comprehensive biological knowledge about NER together with mathematical modelling. In agreement with a simplified model simulating the assembly of a catalytic protein repair-complex \cite{Terstiege2010} we find that reversible exchanging NER factors is the key property that makes NER an apparent first-order process. Accordingly, the long time scale arises from the concurrence of stochastic assembly and reversible exchange, which leads to many non-functional preliminary stages before eventually the catalytic complex is formed. As suggested by Luijsterburg \textit{et al.} (2010) \cite{Luijsterburg2010}, the slow repair time might even present an advantage for the specificity of the repair process due to differential dissociation kinetics in case of a false-positive lesion detection. This is in contrast to a sequence of irreversible binding steps, which will create sigmoidal kinetics with an even sharper delay with increasing numbers of assembling components \cite{Terstiege2010}.\\ 
The phenomenon of rapid protein exchange is a widespread property also found in other chromatin associated processes such as replication, chromatin remodelling or transcription \cite{McNairn2005,Erdel2011,Sonneville2012,Stasevich2011}. Analogous to NER, these proteins are involved in finding the specific reaction site before the catalytic machinery can assemble at the same location. Particularly in case of DNA transcription, transient protein dynamics were experimentally demonstrated for transcription factor binding and transcription machinery assembly \cite{Hager2009}. The time scale for these reactions (several seconds to minutes) is comparable to our results for DNA repair, which encourages us to hypothesise whether the  slow time scales of transcriptional bursting in mammalian cells (tens of minutes to hours; \cite{Harper2011,Suter2011}) also arise through the reversible assembly of large macromolecular complexes.   



\paragraph{An identifiable and hence predictive kinetic model of DNA repair.}
Augmenting the extensive kinetic data of the binding and dissociation of individual components of the NER machinery \cite{Luijsterburg2010} with a direct readout for DNA repair synthesis we were able to develop a predictive model of NER (cf.\ Chapter \ref{chap:kineticNERmodel}). To the best of our knowledge it is the first model of a DNA-associated process, for which the model parameters could be identified (i.e., parameter values uniquely assigned with narrow confidence bounds) from the experimental data. A prerequisite for the computational feasibility of the identifiability analysis (calculating the profile likelihoods for each parameter) is the runtime optimization of the parameter estimation algorithm. Therefore, we integrated the model into a dynamic modelling framework that applies a deterministic trust region algorithm implemented in MATLAB \cite{Raue2009}. Accuracy and computational performance of the implementation benefit immensely from the simultaneous solution of the sensitivity equations \cite{conn2009introduction,Ramachandran2010,Raue2013}, which provide the required derivatives of the objective function with respect to the parameters. In fact, solving this extended system of ordinary differential equations with a CVODE solver implemented in C accelerated the speed for one estimation run by 2x$\text{10}^\text{2}$ fold compared to the previously applied stochastic Markov Chain Monte Carlo (MCMC) optimization algorithm. The runtime spent for one fit on a 12-core processor with 12 GHz each is now reduced to $\sim$10 minutes. This allowed us to sample the parameter space very effectively using Latin hypercube sampling, which makes the deterministic optimization algorithm more robust against local optima \cite{Raue2013}. As well as fitting the parameters that determined the model dynamics, we also estimated the parameters that characterize the measurement noise. This approach has the advantage that it avoids an inaccurate estimate of the experimentally-determined error due to low numbers of replicates and thus gives a more exact assessment of the model parameters, without using a preprocessing of the experimental data \cite{Raue2013}. \\
% * <l.adlung@dkfz.de> 2014-12-10T23:14:52.333Z:
%
%  Do you ever show the step-like function of the sorted X^2 to show that the global optimum was robustly found?
%
To achieve an identifiable NER model, a number of simplifications concerning the model structure were required. These were partly imposed by the limited collection of NER factors, which are currently available for live-cell imaging. In addition, the profile likelihood analysis itself allocated model substructures, which are not sufficiently described by the measurements. Accordingly, the two most significant modifications compared to the work of Luijsterburg \textit{et al.} (2010) \cite{Luijsterburg2010} concern the removal of the 'partially unwound' repair intermediate and the neglect of protein-protein interactions, in particular the sequential binding of XPA and XPF, for which there was no evidence in the data. Due to the lack of a direct readout for incised DNA, the incision of the DNA lesion has been assumed practically instantaneous once the pre-incision complex has been completely assembled. Despite the rate of rechromatinisation, this uncertainty certainly also affects all other catalytic rates, indicated by the existence of lower bounds, only. Consequently, the individual rate constants turn out to be fast, which agrees with a recent direct measurement in bacteria where DNA synthesis and ligation take seconds \cite{Uphoff2013}. \\
The current status, where the model structure is fully reconcilable with the experimental data, denotes a concrete progression beyond previous NER models \cite{Luijsterburg2010,Politi2005,Kesseler2007}. At the same time this enclosed data-model entanglement defines the predictive power of the model and thus determines what can be said quantitatively about the regulation of NER. Consequently, gaining further mechanistic understanding about system properties requires a parallel and intertwined advancement where the addition of molecular detail is consecutively balanced with appropriate quantitative measurements. We expect that the general dynamic behaviour of the model (\textit{e.g.} repair rate, robustness against protein-expression noise and fidelity of lesion recognition) will prove robust with respect to such a development as we observed these properties already for a simplified model of repair (cf.\ Figure \ref{fig:reactionTiming}).    
       

\paragraph{Model parametrization and its implications}
The model fit yields biochemically plausible estimates for the kinetic parameters of the individual assembly and dissociation kinetics \textit{in vivo} that account for both the long-term accumulation and the rapid exchange of the NER factors. Notably, the affinity of the lesion recognition factor XPC is remarkably low, which is mainly caused by the high $k_{\text{off}}$ rate. This is consistent with previous work where high molecular off-rates have been described as a general property of self-organizing systems, which allows for efficient exploration of an assembly landscape and selection of a functional steady state \cite{Kirschner2000}. As discussed in depth at Luijsterburg and coworkers (2010)\cite{Luijsterburg2010} the strong reversibility is also beneficial for the specificity and regulation of the system, for example by preventing the trapping of NER proteins in incomplete (and thus unproductive) repair complexes. The same applies for the trade-off between specificity and efficiency of the mechanism, which equally plays a role in chromatin remodelling and transcription \cite{Cook2010,Voss2011}.\\
The parameter set is complemented by the fast catalytic rate constants, which indicate a tight coupling between lesion excision and repair synthesis. This implies that the lesions are repaired without much delay immediately after their recognition by XPC, which consequently prevents the accumulation of incised DNA. This \textit{in-vivo} finding stays in contrast to \textit{in-vitro} experiments that have found a delay before repair synthesis \cite{Mocquet2008,Riedl2003}. The predicted prevention of the single-stranded repair intermediate corresponds with our intuition that an elevated abundance of incised DNA would trigger DNA degradation to avoid the risk of an inflammatory response or autoimmunity \cite{Takeuchi2010}.             



\paragraph{Collective control of the DNA repair rate}
A consequence of the NER model architecture, expressing the sequential organization of this chromatin-associated process through cycles of protein recruitment, is the absence of a rate-limiting factor for repair synthesis. The control is rather homogeneously distributed with small contributions by each repair factor. This result is similar to the outcome of previous flux control analyses that found no experimental support for the existence of a unique rate-limiting factor \cite{Fell1992}. Moreover, response coefficients with values much smaller than 1 indicate that DNA repair is robust  against natural fluctuation of the protein concentrations in the cell \cite{Bluthgen2013}. It is important to note that the degree of control as a kinetic property of a particular NER factor is independent of the order of engagement in the repair process. Thus, the lesion recognition factor XPC, which binds first, can have the same control as XPA or RPA that bind much later to protect the unwound DNA single-strands.\\
As a biochemical network can never be insensitive against all possible perturbations \cite{Bluthgen2013,Csete2002}, DNA repair will be affected by larger fluctuations in the concentration of NER factors. In this respect, the model predicts that the rate of repair eventually drops to zero in case of a strong concentration reduction of any repair factor. This agrees with various studies showing that a significant reduction in nuclear XPC and XPA levels leads to decreased cell survival and/or decreased lesion removal \cite{Koberle1999,Koberle2006,Renaud2011}. In contrast, increased levels of ERCC1 and XPA lead to increased resistance in certain tumours against cisplatin, which induces DNA interstrand crosslinks specifically removed by NER \cite{Koberle1999,Koberle2006,Renaud2011,Stewart2007,Arora2010}.   
\\ 
To analyse the effect of natural changes in NER factor expression on the repair rate experimentally, we exploited the natural variability of the nuclear protein concentration instead of changing the protein expression by gene knock-down or overexpression, which usually cause the partial or complete disruption of essential system functions \cite{Moriya2006}. Despite the weak correlation in the data we saw a significantly positive dependency between protein expression for all five measured NER factors and DNA resynthesis. The regression slopes were evenly distributed and below unity, which corroborates the finding of a shared moderate repair-rate control.\\  
Not surprisingly, there are studies indicating that the collective control of a systems property (\textit{e.g.} repair rate) is not an exclusive concept for the DNA repair pathway, but also applies to other chromatin-associated processes. For example, it was shown that the probability to express interferon-$\beta$ was significantly increased when five out of six transcription factors, necessary for transcriptional activation, were overexpressed \cite{Apostolou2008}. A very similar result was recently obtained, which suggested the elevated expression of IL-2 by the concerted interplay of four transcription factors together with FOXP3 \cite{Bendfeldt2012}. These findings emphasize the probability of a collective process also for transcription.   

% * <l.adlung@dkfz.de> 2014-12-10T23:28:41.870Z:
%
%  Are those collective overexpression results always compared to single overexpression?
%

\paragraph{Co-regulation of the nuclear NER factor expression}
For the mechanistic interrelation between robustness as a systems property and the corresponding biochemical network several regulatory motifs on the molecular level (\textit{e.g.} negative feedback, incoherent feed-forward loops and functional redundancy) have been proposed \cite{Alon2007,Bleris2011,Yi2000,Manuscript2011}. However, most of these functional building blocks come not into question for promoting robust DNA repair as the NER pathway manages without any reported signalling events (shown in this work), nor should transcriptional regulation play a role on the considered time scale \cite{Mone2001}. The individual repair reactions purely depend on the repair complex assembly time and thus are determined by the NER factor binding affinity (this work and \cite{Luijsterburg2010}). Due to the long, random and reversible, assembly process of the whole repair complex, minor delays or speed-ups in one assembly step caused by small fluctuations of one repair factor are negligible on the global time scale. In this way the robustness of DNA repair is directly linked to the dynamic design of the repair pathway.\\     
Intriguingly, we found experimental evidence for the regulated expression of the NER factors themselves, indicated by a strongly positive pairwise correlation for all seven measured repair factor combinations. In contrast to previous studies showing correlated mRNA levels for different NER factors in cancers cells \cite{Damia1998,Cheng2000} we observe this cross-correlated expression in single cells of different type under unperturbed, NER-independent conditions. Despite a general increase in the nuclear NER factor concentration during the S-phase and G2 phase, the cross-correlation persists independently of the cell cycle. Together, these results imply an additional regulative motive that indirectly controls the repair rate already on the transcriptional level.\\
Remarkably, the co-expression of NER factors has strong conceptual similarity to bacterial signal transduction systems where different regulatory enzymes are encoded on the same operon \cite{Kollmann2005,Lovdok2009}. This co-localisation leads to co-expression, which in turn provides concentration robustness against variability of other pathway components \cite{Bluthgen2013}. So far such a transcriptional coupling mechanism is unknown for the repair proteins involved in NER. It would be therefore interesting to investigate whether this co-expression can be perturbed pre-transcriptionally to further study its impact on the rate of repair. 

% * <l.adlung@dkfz.de> 2014-12-10T23:33:35.858Z:
%
%  Perturbation in the sence of mono-cistronic mRNA?
%
% ^ <l.adlung@dkfz.de> 2014-12-10T23:36:20.513Z:
%
%  Is there any RNAseq data to corroborate correlated mRNA levels for the NER factors?
%

\paragraph{Concluding remarks}
The accurate repair of the nuclear DNA is pivotal for the enduring vitality of the cell and in particular for the fidelity of all chromatin-associated processes. It is therefore not surprising to find two cooperating molecular approaches that ensure robustness of the repair rate against natural variability. First, as a consequence to the dynamic design of the repair complex assembly there is only a moderate and uniformly distributed control of the repair rate by each individual NER factor, which translates into a low system response in the event of NER factor fluctuations. Second, a so far unknown extrinsic mechanism regulates the coordinated NER factor expression presumably on the transcriptional level, which limits the degree of protein variability already pre-transcriptionally. Together both mechanisms might be a widespread mode of robust regulation of chromatin-associated processes, given that chromatin remodelling, transcription and translation have similar dynamic features as DNA repair.