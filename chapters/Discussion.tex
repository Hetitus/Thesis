\chapter{Discussion}


Our continuation of the DNA-repair analysis was motivated by the ability of explicitly observing the kinetics of resynthesized DNA during nucleotide-excision repair. The quantification of this repair intermediate by following the incorporation of fluorescently labeled nucleotides portrays the repair process as a slow, first-order kinetic. The new data supplement our comprehensive knowledge about the dynamic behavior of single repair proteins comprising their individual exchange kinetics at the chromatin fiber, which eventually lead to the formation of catalytic multi-protein machineries. Based on this information we were able to identify a realistic kinetic model of NER, which established a quantitative link between fast and random assembly of the NER factors at the DNA template and the emergent phenomenon of a slow overall repair time. Most importantly, the model predicts a collective control of the repair rate by the repair factors, which rebuts the presumption of a rate-limiting step in the pathway. This finding has been corroborated experimentally exploiting the natural variability of the nuclear NER factor concentrations, which highlights the robustness of DNA repair against fluctuations in NER factor expression.\\       
Correlation analysis revealed 

on the single-cell level

The quantitative impact of these three parameters on the cell-to-cell variability of the repair rate is consistent with the robustness of the repair rate against concentration fluctuations in individual repair factors.


\paragraph{Slow first order kinetics as a systems property}



\paragraph{First identifiable kinetic model of DNA repair}
- no partially unwound DNA
- no XPA before XPF binding
- comparison to previous models 


\paragraph{Model predictions}
- tight coupling between damage removal and repair synthesis
- little accumulation of incised DNA


\paragraph{Rate control}
- lesions are far from saturation 
- rate control is a kinetic property - independent of the order of binding
- similarity to the control of fluxes, which are known to share control...
- implications for transcription 
- experimental validation
- difficulties during correlation analysis
- further rate control through cooperativity 


\paragraph{cross-correlation of nuclear expressed proteins}
- interpretation of the data
- implications for less correlated systems
- so far no correlation measured on the protein levels 
- only mRNA level - we don't see that
- it would be interesting to see what happens under the influence of drugs...

