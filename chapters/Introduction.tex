\chapter{Introduction}
\pagenumbering{arabic}
\pagestyle{plain}

\section{DNA damage and its consequences}

The impeccable preservation of DNA is surely one of the most essential requirements for cellular viability. On a single day DNA accumulates on average $\sim\text{10}^\text{5}$ damages per cell, which can be grouped into three different categories \cite{Hoeijmakers2009}. First, DNA constantly undergoes spontaneous unspecific reactions (mostly hydrolysis) that cause deamination and create abasic sites in an aqueous solution \cite{Lindahl1993,Lhomme1999}. Second, the most common type of DNA damage originates from the cellular metabolism, which produces various types of reactive oxygen and nitrogen species,lipid peroxidation products, endogenous alkylating agents, estrogen and cholesterol metabolites, and reactive carbonyl species. The subsequent molecular distortions range from several kinds of single-strand breaks to various oxidative base and sugar products \cite{DeBont2004,Sander2005}. And third, DNA damages that are caused by external physical or chemical agents like sunlight, which induces approximately 30,000 pyrimidine dimers per hour in each exposed keratinocyte \cite{Luijsterburg2010}.\\        
The consequences for cellular viability that arise from the diversity of DNA lesions are essentially twofold:

either mutagenic or cytotoxic effects    

    

 



- Transcription relies on an intact template (Martijn 2014)
- DNA damage cannot be replaced but has to be repaired (Martijn 2014)
- influence on ageing and cancer (Martijn 2014)
- repair mechanisms need to be robust...
- genome instability - a hallmark of cancer (Bartek 2007)



\subsection{Variety of DNA damage repair mechanisms}
- DNA damage response
-The repair systems include the direct reversal of
damage, nucleotide excision repair (NER), base excision repair (BER), DSB repair, and interstrand crosslink repair (Hoijmarkers 2001)

\section{Mammalian nucleotide excision repair}
\label{sec:NERexperiments}
- Global genome and transcription coupled NER

\paragraph{Key components of the NER pathway}
\paragraph{NER as a model system for chromatin-associated processes}
comparison between CPDs and 6-4PP

\section{NER as model system for chromatin associated processes}

