\chapter{Introduction}
\pagenumbering{arabic}
\pagestyle{plain}

\section{DNA damage and its consequences}

The impeccable preservation of DNA is surely one of the most essential requirements for cellular viability. On a single day DNA accumulates on average $\sim\text{10}^\text{5}$ damages per cell, which can be grouped into three different categories \cite{Hoeijmakers2009}. First, DNA constantly undergoes spontaneous unspecific reactions (mostly hydrolysis) that cause deamination and create abasic sites in an aqueous solution \cite{Lindahl1993,Lhomme1999}. Second, the most common type of DNA damage originates from the cellular metabolism, which produces various types of reactive oxygen and nitrogen species,lipid peroxidation products, endogenous alkylating agents, estrogen and cholesterol metabolites, and reactive carbonyl species. The subsequent molecular distortions range from several kinds of single-strand breaks to various oxidative base and sugar products \cite{DeBont2004,Sander2005}. And third, DNA damages that are caused by external physical or chemical agents like sunlight, which induces approximately 30,000 pyrimidine dimers per hour in each exposed keratinocyte \cite{Luijsterburg2010}.\\        
The consequences for the cellular viability that arise from the diversity of DNA lesions are essentially twofold. Either, the inflicted chromosomal aberrations are not recognized or bypassed by the DNA repair machinery and thus persist in the genetic Code \cite{Hoeijmakers2009}. These mutations can activate oncogenes or inactivate tumour suppressor genes, which is equivalent to an increase in the cell's cancer risk level \cite{Bartek2007}. Or, alternatively, the damage is so strong that essential cellular processes such as transcription or replication are impaired. This is usually the case for interstrand cross-links or double-strand breaks induced by ionizing irradiation. The occurrence of such lesions leads to accelerated cell death, which particularly in proliferative tissues promotes symptoms of premature ageing \cite{Marteijn2014}. Depending on the location and number of lesions, the cell type, stage in the cell cycle and differentiation both cell fates can happen concurrently. In fact, both outcomes are directly linked considering that damage-induced cell-death protects the body from cancer \cite{Hoeijmakers2009}.            

\paragraph{DNA damage response}    
DNA is the only known biological macromolecule, which cannot be replaced if parts of it are destroyed. This has the inevitable consequence that any mutation that escapes cell-death will persist over the lifetime of the cell \cite{Hoeijmakers2009,Marteijn2014}. For this reason, cells evolved an elaborate genomic maintenance apparatus, the DNA damage response (DDR), that keeps DNA damage under control \cite{Ciccia2010}. The DDR  
has to do many decisions: prophylaxis through tanning; apoptosis - senescence; which type of



- mismatch repair (MMR) (Jiricni 2006)
- (BER) - small chemical alterations (Lindahl 2000)
- ICL (inter cross link lesions) - Moldovan 2009
- NER - Mertijn 2014
- NHEJ/HR  West 2003
- DNA damage response has to do many decisions: prophylaxis through tanning; apoptosis - senescence; which type of repair (Ciccia 2010)
- p53 regulates a lot (apoptosis, senescence, cell-cycle arrest) (Riley 2008)


- DNA damage cannot be replaced but has to be repaired (Martijn 2014)

- repair mechanisms need to be robust...

 
- DNA damage response
-The repair systems include the direct reversal of
damage, nucleotide excision repair (NER), base excision repair (BER), DSB repair, and interstrand crosslink repair (Hoijmarkers 2001)

\section{Mammalian nucleotide excision repair}
\label{sec:NERexperiments}
- Global genome and transcription coupled NER

\paragraph{Key components of the NER pathway}
\paragraph{NER as a model system for chromatin-associated processes}
comparison between CPDs and 6-4PP

\section{NER as model system for chromatin associated processes}

