\chapter*{Acknowledgments}

\addcontentsline{toc}{chapter}{Acknowledgments}

The present thesis was carried out in the Division of Theoretical Systems Biology headed by 
Prof. Dr. Thomas H\"ofer at the German Cancer Research Center (DKFZ) and BioQuant Center, 
University of Heidelberg. At this point I would like to thank everyone who has made this work 
possible and who has directly or indirectly supported me during my Ph.D. 

Mein gr\"o\ss ter Dank gilt Prof.\ Dr.\ Thomas H\"ofer f\"{u}r die Annahme als Doktorandin in 
seiner Arbeitsgruppe, die Bereitstellung dieses faszinierenden Themas und insbesondere f\"{u}r seine umfangreiche 
Betreuung. 
%Thomas H\"ofer hat es mir ermöglicht von der theoretischen Mathematik in die 
%Systembiologie zu wechseln  
Durch sein enormes Fachwissen, seine st\"{a}ndige Bereitschaft zur Diskussion und seine 
unsch\"{a}tzbar wertvollen Ideen war er mir eine gro\ss e Hilfe und ich durfte sehr viel von ihm 
lernen. Dar\"{u}ber hinaus danke ich ihm, dass er mir die Teilnahme an vielen 
nationalen sowie internationalen Konferenzen erm\"{o}glicht hat um meine Arbeit dort zu 
pr\"{a}sentieren und wichtige Erfahrungen zu sammeln.

Ganz besonders m\"{o}chte ich mich bei Prof.\ Dr.\ Anna Marciniak-Czochra daf\"{u}r 
bedanken, 
dass sie die Entwicklung meiner Arbeit im Laufe der Jahre mit motivierendem Interesse 
verfolgt 
hat.
Ihre hilfreichen Anregungen und Ratschl\"{a}ge waren stets eine fachliche Bereicherung. 
Dar\"{u}ber hinaus danke ich ihr f\"{u}r die \"{U}bernahme des 
Erstgutachtens.

Ein herzlicher Dank geht an Prof.\ Dr.\ Ursula Klingm\"{u}ller f\"{u}r ihr konstruktives Engagement in meinem Ph.D.-Komitee. Ihre gezielte Fragestellungen und wertvollen Kommentare haben 
unsere Diskussionen wesentlich vorangebracht. Zudem m\"{o}chte ich mich bei ihr daf\"{u}r 
bedanken, dass sie an meiner Verteidigung als Pr\"{u}ferin agiert.

Besonders danke ich Prof.\ Dr.\ Dr.\ h.c.\ mult.\ Hans Georg Bock f\"{u}r sein Mitwirken in meiner Pr\"{u}fungskommission. Seine Begeisterung f\"{u}r die angewandte Mathematik hat mich bereits zu Beginn meiner Doktorandenzeit motiviert. 
%TODO change examiner?

Ein herzliches Dankesch\"{o}n geht an meinen Kollegen Dr.\ Michael Flo\ss dorf, welcher mich besonders in der Anfangszeit mitbetreut hat und meine Arbeit durch viele wertvolle Ideen voranbrachte. Speziell danke ich ihm, dass er sich stets f\"{u}r meine Fragen Zeit genommen hat und er mir durch sein enormes Wissen sehr viel beibrachte.
%Durch sein enormes Wissen durfte ich sehr viel von ihm lernen.

Daneben geht ein ganz besonderes Dankesch\"{o}n an unsere experimentellen Kooperationspartner Dr.\ Ulfert Rand, Dr.\ Mario K\"{o}ster und Dr.\ Hansj\"{o}rg Hauser. Vor allem danke ich ihnen, dass sie mir ihre herausragenden Resultate anvertraut haben und mich mit viel Geduld dabei unterst\"{u}tzten die Biologie hinter den Daten zu verstehen. An meinen ``wet lab''-Schnupperkurs werde ich stets mit einem L\"{a}cheln zur\"{u}ckdenken. 

Dar\"{u}ber hinaus danke ich Dr.\ Bianca Schmid und Prof.\ Dr.\ Ralf Bartenschlager f\"{u}r eine spannende Kollaboration, in welcher ich mittels ihrer \"{u}berragenden Daten und wertvollen Diskussionen in das faszinierende Gebiet der Ausbreitung von Dengue Viren eintauchen durfte. An dieser Stelle m\"{o}chte ich Bianca ganz besonders daf\"{u}r danken, dass sie mir wissenschaftlich und auch privat immer zur Seite stand. 

% our working group
I am very thankful to the current and former members of the H\"ofer group for the pleasant working atmosphere, which has given me solid support and encouragement in my work. Special thanks go to Krist\'{o}f K\'{a}ly-Kullai for his expert assistance during the initial phase of my Ph.D. 

% Nils, Tim, Erika, Anna,...
Speziell danke ich Dr.\ Nils Becker, Tim Heinemann, Dr.\ Erika Kuchen sowie Dr.\ Anna Schulze f\"{u}r das kritische Lesen meiner Doktorarbeit und die hieraus enstandenen konstruktiven Diskussionen. 


%mir mit viel Geduld dabei halfen die Biologie hinter den Daten zu verstehen. 
%Alle Experimente im 2. Kapitel dieser Arbeit wurden 
%cooperation partners Ulfert Rand, Mario K ̈oster and Hansj ̈org Hauser from the Department of Gene Regulation and Differentiation at the Helmholtz Centre for Infection Research (HZI) in Braunschweig 

%Ein Dankesch\"{o}n 



%
%und mein Forschungsprojekt durch ihre Ideen, ihre Anregungen und ihre konstruktive Kritik bereicherten.

%
%
%
%Allen Kolleginnen und Kollegen, die ich “daheim” im Arbeitsbereich Numerische Mathematik und auf den Dienstreisen kennenlernen durfte, und besonders meiner Familie danke ich fu ̈r die Unterstu ̈tzung.
%Diese Dissertation wurde von der DFG im Projekt LU 532/4-1 gef ̈ordert, was maßge- blich zu den hervorragenden Arbeitsbedingungen und Reisem ̈oglichkeiten beigetragen hat.
%

% Thomas (change from theoretic to systems biology, great topic, encouraging with a smile, talks larger conferences, travelling, computer, cluster)

% Anna Marciniak
% Ursula Klingmüller

% 4. examiner

% Hansjörg Hauser
% Mario Köster
% Ulfert Rand

% Ralf Bartenschlager
% Bianca Schmid

% Michael Flossdorf 
% Kristo´f Ka´ ly-Kullai

% our working group
% Nils, Tim, Erika, Anna,...


% --------------------------------------------------------------------------------------------------------

% Diana Haendly

% Michael Nemetz
% Nick Kepper 

% Jonas Förster

%This work was supported in part
%by
%DKFZ
% ImmunoQuant (Anne Weiser, Miriam)
% grants from the German Research Foundation (SFB 900), the BMBF
%MedSys and ForSys initiative (ViroQuant), and the Initiative and
%Networking Fund of the Helmholtz Association within the Helmholtz
%Alliance on Systems Biology/SBCancer.

% Rainer Nagel
% Tanja Eisner

%Gabriele Gauckler
% Ludwig Gauckler

% Mum
% Grandmum




% ---------------------------------------------------------------------
% Karl Rohr
% Nathalie Harder
% Jan-Philip Bergeest

% working group in Braunschweig Johannes Schwerk, Gesa Nöhren, Melanie Linnes, Andrea Kröger

% working group of Ralf Bartenschlager, especially Marco Binder, Alessia Ruggieri, Antje Reuter, Tim Hart, Wolfgang Fischl

% Math teacher

% BioQuant cleaning service 